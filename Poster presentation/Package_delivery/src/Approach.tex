\justify
\begin{itemize}
\item{Data Collection - In order to simulate a more realistic simulation environment we collected terrain, population, Walmart, and K-12 schools from San Jose. The data was provided from the United States Geological Survey (USGS), United States Census Bureau, Walmart.com, and Schooldigger.com, respectively.}


\item{Initial Framework - The structure of the simulation focused around which warehouses were chosen to have a delivery fleet. Elevation and obstacles around these warehouses are compiled before the simulation runs.}


\item{Single Package Delivery - UAS vehicles reponded to one package request at a time traversing to the destination and following the same path back to the warehouse.}


\item{Multiple Package Delivery - UAS vehicles carrying multiple packages optimize which destinations to delivery to based on proximity to the first destination. After being assigned the first package to deliver, the vehicle waits on the ground until a nearby second package is requested or if the time to deliver is at the time constraint.}


\item{Analysis - To assess the feasibility of a multiple package per vehicle delivery system, we ran the simulation using both the single and multiple package per vehicle approach comparing the average flight distance, total number of packages delivered, and so as to compare how the vehicle would perform.}
\end{itemize}
