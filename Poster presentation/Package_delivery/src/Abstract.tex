\textbf{\textit{Quadcopters are being tested to perform a variety of maneuvers while utilizing different controller models, and the results have shown a promising future. For a few years now, delivery companies have been developing a variety of package-delivery systems using drones. Such technologies will most likely be integrated and implemented into fully functioning autonomous Unmanned Aerial Systems (UAS) in the near future. However, the lack of studies on the effect of such a delivery system on the current National Airspace (NAS) traffic is unstudied.  Moreover, storage and performance requirements to implement a fully autonomous delivery system have not yet been quantified. Due to this reason the production and feasibility of the system is hindered. In this project we implemented a test bench to  that uses terrain, building, and population data to build an environment in which a package delivery scenario may be run using quadcopters.  Terrain information was used to create a more realistic scenario which the quadcopters will be used in. Population density information from census bureau around the warehouses was used to better estimate when, where, and how big of a package will be ordered from the warehouse.  We showed that how number of vehicle per warehouse, how soon the warehouse guarantee the package to be delivered, and on board storage required to run the system} }

%This statement requires citation \cite{Smith:2012qr}.
