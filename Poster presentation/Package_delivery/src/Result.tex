
For both 8 and 4 hours of simulated delivery time, the multiple package delivery system out performed the single package delivery system in overall performance and cost for the delivery service. 
{\footnotesize
\begin{table}[]
\centering
\caption{Simulation Results - Comparing single and multiple packages per vehicle with a 25 vehicle fleet}
\label{my-label}
\begin{tabular}{|l|l|l|l|}
\hline
\begin{tabular}[c]{@{}l@{}}Packages Per\\ Vehicle\end{tabular} & Simulation Time & \begin{tabular}[c]{@{}l@{}}Average Distance\\ Traveled\end{tabular} & Packages Delivered \\ \hline
One-Package                                                    & 8 Hours         & 120 Km    & 121    \\ \hline
One-Package                                                    & 4 Hours         &  67.07 Km & 77     \\ \hline
Two-Packages                                                   & 8 Hours        & 101 Km     & 127    \\ \hline
Two-Packages                                                    & 4 Hours         &  64.4 Km   & 81     \\ \hline
\end{tabular}
\end{table}
}
\vspace{3mm}

By showing that more packages can be delivered when multiple packages are transported by one vehicle, we prove that it would be more cost efficient for a warehouse to invest in vehicles with larger payload capacity. Our simulations also showed that by using the right logic for which packages are assigned to which vehicle, we can ensure that deliveries are made within the time constraint set by the warehouse. We have proved the multiple package delivery system was successful since we delivered more packages using the same resources while reducing the average distance each vehicle had to traverse.
