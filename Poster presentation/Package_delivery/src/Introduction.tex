% template example
%Lorem ipsum dolor \textbf{sit amet}, consectetur adipiscing elit. Sed commodo molestie porta. Sed ultrices scelerisque sapien ac commodo. Donec ut volutpat elit. Sed laoreet accumsan mattis. Integer sapien tellus, auctor ac blandit eget, sollicitudin vitae lorem. Praesent dictum tempor pulvinar. Suspendisse potenti. Sed tincidunt varius ipsum, et porta nulla suscipit et. Etiam congue bibendum felis, ac dictum augue cursus a. \textbf{Donec} magna eros, iaculis sit amet placerat quis, laoreet id est. In ut orci purus, interdum ornare nibh. Pellentesque pulvinar, nibh ac malesuada accumsan, urna nunc convallis tortor, ac vehicula nulla tellus eget nulla. Nullam lectus tortor, \textit{consequat tempor hendrerit} quis, vestibulum in diam. Maecenas sed diam augue.



One of the main economic and environmental concerns for package delivery arises within the last mile of transit. Often times deliveries are delayed at the warehouse so they may be grouped together to fill a truck's capacity. The truck is then responsible for delivering its entire payload by exhaustively traversing to each customer's house. This is not always the most efficient method and does not guarantee deliveries to be made in the least amount of time. 
Companies are exploring the use of small unmanned aerial systems (UAS) for local package deliveries in an attempt to minimize these inefficiencies and to offer rapid delivery services \cite{Davidson:2013web}. However, the effect this would have on the National Airspace (NAS) is mostly unstudied. In an attempt to study the traffic that would be produced, we aim to build a simulation that uses UAS vehicles flying at low altitude to deliver packages from local warehouses when requests are made. 
As we build this simulation we must assume certain parameters especially since this type of service is not yet offered. For the purposes of analyzing the air traffic on a larger scale, we will assume that the warehouse will manage ground level traffic for takeoff and landing purposes at the warehouse. When using small UAS vehicles, it is safe to assume this will not cause much change in the overall traffic of the simulation.
Another assumption we will make for the early stages of development is that all warehouses are run independently of each other. This will allow us to focus on fully developing the traffic management around one warehouse at a time. Each warehouse will have its own fleet of UAS vehicles and interact with the local environment around it. For further development of the simulation, warehouses will need to communicate between each other so that traffic between two adjacent warehouses can be optimized in a way that meets demands of package requests around both warehouses. This assumption currently limits each warehouse to have nonoverlapping areas of delivery. Since UAS vehicles from different warehouses do not currently know the position of other UAS vehicles from other warehouses, collisions would not be avoided. 
